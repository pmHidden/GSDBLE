\chapter{Exposé}

\section{Ausgangslage und Zielsetzung}
Drohnen können zukünftig eine große Rolle in der regionalen Infrastruktur spielen. Sie werden meist elektrisch betrieben und belasten damit nicht die lokale Luftverschmutzung. Sie können sich schnell durch die Luft bewegen und strapazieren nicht den Straßenverkehr. Sie können kleinere Lasten transportieren und eignen sich somit wunderbar zum Kleintransport innerhalb von Städten. So möchte Amazon zukünftig Pakete per Luftpost verschicken\footnote{\url{https://www.tagesschau.de/wirtschaft/boerse/amazon-drohne-101.html}}. Auch eignen sich Drohnen zur Erkundung von Gebieten. Die Feuerwehr testet sie zur Einschätzung von Bränden\footnote{\url{https://www.digitalstadt-darmstadt.de/drohnen-feuerwehr/}}.\\
Bei allen Szenarien ist es sinnvoll mit der Drohne kommunizieren zu können. Hierbei ist in der Frequenz des Funkkanals zu unterschieden. Eine höhere Frequenz bedeutet mehr Datendurchsatz und eine Geringere mehr Reichweite. Handelsübliche Drohnen übertragen dabei meist über die Frequenzen 2,4 - 2,4835 GHz. Diese schaffen unter besten Bedingungen etwa 2 km\footnote{\url{https://www.dji.com/de/spark/info\#specs}}. Ziel dieser Arbeit ist es, mithilfe eines LoRa-Moduls, das auf einer Frequenz von 868 MHz sendet, einen Kommunikationskanal aufzubauen und zu evaluieren.

\section{Methoden und Vorgehen}
Die zur Verfügung gestellte Drohne ist eine Intel Aero RTF. An diese soll ein LoRa-Modul angeschlossen werden. Dafür verfügt die Drohne über 15 GPIOs, die über ein Linux Betriebssystem ansprechbar sind. Da das LoRa-Modul für eine Arduino-Pinbelegung gedacht ist, muss untersucht werden, ob der Anschluss über diese GPIOs möglich ist. Sollte dies nicht funktionieren, muss das Modul an einen Arduino angeschlossen werden und der Arduino an die Drohne. Dann müsste sowohl für den Arduino als auch für das Linux ein Programm geschrieben werden. Dies bedeutet mehr Gewicht, Latenzen und Energieverbrauch. Die Webseite der Drohne schreibt: "A custom-built I/O expansion board, 34 GPIOs (3.3 V), 5 analog inputs (0 to 3 V), 1 HSUART (shared), and 1 CAN bus are available."\footnote{\url{https://software.intel.com/en-us/aero/compute-board}}. Es müsste geprüft werden, ob dies auch eine Möglichkeit wäre, da es der Pinbelegung des Arduino ähneln könnte. Es ist auf der Webseite sonst nichts über das Expansion Board zu finden.\\
Auf der anderen Kommunikationsseite wird ein Arduino mit einem weiteren LoRa-Modul gestellt. Diese können dann an ein beliebiges Gerät angeschlossen und über eine serielle Konsole gesteuert werden.\\
Bei der Programmierung stellen sich technische Herausforderungen wie die API der Drohne um zum Beispiel an Sensordaten zu kommen oder die LoRa-Module richtig anzusprechen. Auf der anderen Seite müssen Gesetze und Bedingungen für die Funkkommunikation eingehalten werden\footnote{\url{https://www.bundesnetzagentur.de/SharedDocs/Downloads/DE/Sachgebiete/Telekommunikation/Unternehmen_Institutionen/Frequenzen/Allgemeinzuteilungen/2018_05_SRD_pdf.pdf;jsessionid=F337AE0C6953C29A4C1E7EA233CD0C52?__blob=publicationFile&v=2}}. Zunächst aber kann sich die direkte Verkabelung von Drohne und LoRa-Modul als schwierig gestalten.\\
Um die Kommunikation zu Evaluieren müssen Parameter gefunden werden. Dafür kämen unter anderem Reichweite, Datendurchsatz und Energieverbrauch infrage. Diese sind aber nicht frei von Problemen. So hängt der Datendurchsatz zum Beispiel von der Kanalbreite und anderen Geräten auf der gleichen Frequenz ab. Die Reichweite bedingt sich durch die verwendete Antenne und der Umgebung.

\section{Erwartete Ergebnisse}
Es ist zu erwarten, dass die Reichweite wesentlich mehr als 2 km wie auf der 2,4 GHz Frequenz beträgt. Der Datendurchsatz sollte ausreichen um Sensordaten zu übertragen und könnte auch ausreichen, um die Drohne zu steuern. Bildübertragung wäre für praktische Szenarien interessant und bei sehr kleiner Auflösung vielleicht möglich.