%!TEX root = ../tudkom_students__201804_v1.4.tex
\begin{abstract}
In dieser Arbeit wird ein Wearable entwickelt, das eine mobile Datenerfassung zur Ganzkörperbewegungserkennung möglich macht.
Im Gegensatz zu den bereits genutzten Systemen in der Unterhaltungselektronik, Filmindustrie und Gesundheitsbranche verzichtet es auf Kabel, soll lange von einer Batterie mit Spannung versorgt und durch eine geringe Größe unterwegs genutzt werden.
Gleichzeitig soll der Aufwand gering sein, die Komponenten zu ersetzen, sodass sich das Wearable für verschiedene Anwendungen anpassen lässt.\\
Es werden verschiedene Komponenten verglichen und ein Prototyp entworfen, auf dem zwei unterschiedliche Sensoren implementiert werden.
Die Daten der Sensoren sollen zu einem Android Smartphone übertragen und graphisch dargestellt werden.
Im Anschluss wird ein Wearable aus den Komponenten gebaut, das auch mobil eingesetzt werden kann.
Der Prototyp und das Wearable werden anhand von Stromaufnahme und Schwankungen in der Datenrate bei unterschiedlichen Parametern evaluiert.
\end{abstract}
