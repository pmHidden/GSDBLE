%*****************************************
\chapter{Implementation}
\label{ch:implementation}
%*****************************************

\hint{This chapter should describe the details of the implementation addressing the following questions: \\ \\
1. What are the design decisions made? \\
2. What is the environment the approach is developed in? \\
3. How are components mapped to classes of the source code? \\
4. How do the components interact with each other?  \\
5. What are limitations of the implementation? \\ \\
The section should have a length of about five pages.}

\section{Architecture}
%Wahl der IDE für nRF53832. Offizielle Liste ist hier:
%https://www.nordicsemi.com/DocLib/Content/User_Guides/getting_started/latest/UG/common/nordic_tools
%- IAR: 30 Tage Demo oder beschränkte Funktion (32 Kbyte code u.A.)
%- Keil uVision: beschränkte Version (32 KByte Code und Debugger). Essential Version 1330€ pro Jahr.
%- MBed.org: sehr einfach, da online. code, compiler, bin-datei auf stick ziehen, fertig. leider noch kein mesh support
%- Project Zephyr: 
%- GCC. Diese Anleitung: https://www.disk91.com/2017/technology/hardware/discover-nordic-semi-nrf52832/
%Bug in aktueller GNU Arm Embedded Toolchain 8-2018-q4-major beim Kompilieren: C:/nrf52/arm_tools/bin/arm-none-eabi-objcopy: _build/nrf52832_xxaa.hex 64-bit address 0x4b4fa300000000 out of range for Intel Hex file
%fix: bin/arm-none-eabi-objcopy.exe mit der datei von version 7-2018-q2-update ersetzen.
%Der letzte Part ist einfach mist, weil jeder Ordner einzeln eingetragen werden muss. Am ende bekommt man übrigens die .hex datei und muss die selber rüberkopieren.
%- Segger Embedded Studio: ok, aber meh

\section{Design Decisions}


\section{Interaction of Components}

\section{Summary}