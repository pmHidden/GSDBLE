%!TEX root = ../tudkom_students__201804_v1.4.tex
\chapter{Implementation}
\label{ch:implementation}
%This chapter should describe the details of the implementation addressing the following questions:
%1. What are the design decisions made?
%2. What is the environment the approach is developed in?
%3. How are components mapped to classes of the source code?
%4. How do the components interact with each other?
%5. What are limitations of the implementation?
% 5 pages

\section{Architecture}

\subsection{nRF52832 Software}
Offizielle Liste ist hier:
\url{https://www.nordicsemi.com/DocLib/Content/User_Guides/getting_started/latest/UG/common/nordic_tools}
\begin{itemize}
\item IAR: 30 Tage Demo oder beschränkte Funktion (32 Kbyte code u.A.)
\item Keil uVision: beschränkte Version (32 KByte Code und Debugger). Essential Version 1330€ pro Jahr.
\item MBed.org: sehr einfach, da online. weitere abstraktion wie bei arduino. code, compiler, bin-datei auf stick ziehen, fertig. leider (noch) kein mesh support.
\item Project Zephyr: ist noch nicht offiziell supportet
\item GCC. Diese Anleitung: \url{https://www.disk91.com/2017/technology/hardware/discover-nordic-semi-nrf52832/}\\
Bug in aktueller GNU Arm Embedded Toolchain 8-2018-q4-major beim Kompilieren: C:/nrf52/arm\_tools/bin/arm-none-eabi-objcopy: \_build/nrf52832\_xxaa.hex 64-bit address 0x4b4fa300000000 out of range for Intel Hex file\\
fix: bin/arm-none-eabi-objcopy.exe mit der datei von version 7-2018-q2-update ersetzen.\\
Wenns läuft ists leider trotzdem mist, weil jeder Ordner von der library einzeln eingetragen werden muss. Am ende bekommt man die .hex datei und muss die rüberkopieren.
\item Segger Embedded Studio: funktioniert zwar ganz einfach, aber ist nicht so geil wie eclipse mit zb themes und code folding
\end{itemize}


% folgendes vielleicht zu design
\section{Design Decisions}

notizen:

beim loggen mit rtt gehen gerne pakete verloren. als lösung uart nutzen und mit putty den log lesen.

problem: das debug terminal zeigt den log unsichtbar an.
lösung: nutze ses v4.12: \url{https://devzone.nordicsemi.com/f/nordic-q-a/45985/nrf_log-not-working-on-segger-embedded-studio}

problem: entgegen dem datenblatt nimmt die funktion 'lsm6dsl\textunderscore fifo\textunderscore raw\textunderscore data\textunderscore get' eine 8-bit länge, womit nicht die 4kb der fifo direkt gelesen werden können.
lösung:
die implementierung der funktion ruft nur die funktion 'lsm6dsl\textunderscore read\textunderscore reg' auf, welche eine 16-bit länge nimmt. deswegen kann man hier einfach 'lsm6dsl\textunderscore read\textunderscore reg' mit 16-bit nutzen.

problem: spi maximal 255 bytes lesen in einer transaction wegen 8bit buffer.
lösungen:
- größerer chip hat 16bit buffer, dafür mehr gesamtenergieaufnahme
- buffer manuell im sdk auf 16bit stellen funktioniert, ist aber gegen die hardwarespezifikation aus dem datenblatt
- in einer übertragung mehrere transactions, was viel mehr codeaufwand bedeutet
- mehrere transactions hintereinander



\section{Interaction of Components}

\section{Summary}
