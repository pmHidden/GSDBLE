%!TEX root = ../tudkom_students__201804_v1.4.tex
\chapter{Abschluss}
\label{ch:closure}
%This chapter should summarize the thesis and describe the main contributions of the thesis. Subsequently, it should describe possible future work in the context of the thesis. What are limitations of the developed solutions? Which things can be improved?
% 3 pages

\section{Zusammenfassung}


\section{Zukünftige Arbeiten}
Um die Messmethode zu verbessern, muss das Testobjekt eine feste Spannung bekommen.
Eine Lösung ist es die Widerstände so zu wählen, dass der Spannungsabfall bei allen Messversuchen gleich bleibt.
Beim Advertisen in Abbildung \ref{fig:curCr} sieht man, dass der Strom sich aber über eine breite Spanne ändert.
Idealerweise müsste sich der Widerstand dynamisch ändern und der aktuelle Wert zur Berechnung des Stroms gespeichert werden.
Eine andere Lösung ist eine Spannungsquelle, die so eingestellt werden kann, dass die Spannung hinter den Widerständen gehalten wird.
Um Lastunterschiede auszugleichen ohne kurzzeitig Überspannungen zu erzeugen, müsste die Spannungsquelle sehr schnell reagieren.
Dagegen kann ein Kondensator diese Lastunterschiede abfedern und die Anforderungen an die Spannungsquelle mindern.\\


energieverbrauch beim smartphone

andere unkonventionelle arten zur datenübertragung untersuchen. durch vibrierende knochen reden.

btmesh implementieren und vergleichen, wie stark der energieverbrauch sich ändert. es gibt da die library von nordic

nRF52811

project zephyr oder zukünftig mbed

list feature für easydma

mix aus daten verwerfen und sensor langsamer machen

eInk um verbindung anzuzeigen
