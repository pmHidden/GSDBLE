%!TEX root = ../tudkom_students__201804_v1.4.tex

%This chapter should give a comprehensive overview on the related work done by other authors followed by an analysis why the existing related work is not capable of solving the problem described in the introduction.
% 3-5 pages

\chapter{Aktueller Stand der Technik}

\section{Bestehende Lösungen}
Einige bestehende Hardwarelösungen eignen sich bereits für die Anwendungsfälle.

\subsection{Texas Instruments CC2650STK}
Darunter fällt beispielsweise das TI CC2650STK.
Es besteht aus der CC2650 MCU sowie 10 Sensoren, darunter eine MPU-9250 IMU, die über I2C angeschlossen ist.
Eine Auflistung der überflüssigen Komponenten, die nicht komplett abgeschaltet werden können, findet sich in Tabelle \ref{tab:cmp_cc2650stk}.
Eine CR2032 Knopfzelle mit 230 mAh verliert dadurch $\cfrac{365d*24h}{(230mAh/0.00351mA)} \approx 13.4$ \% der Kapazität pro Jahr.
Die IMU verbraucht 3.4 mA, wenn Gyrosensor bei 1 kHz und Beschleunigungssensor bei 4 kHz laufen \cite{datasheet_mpu9250}.
Die MCU unterstützt zwar BLE, aber kein BT Mesh.
Das System ist nicht mehr in Europa verfügbar. \cite{site_cc2650stk}

\begin{minipage}{\linewidth}
	\captionof{table}{Stromaufnahme der ausgeschalteten zusätzlichen Komponenten am TI CC2650STK}
	\label{tab:cmp_cc2650stk}
	\begin{tabularx}{\linewidth}{X|X}
    Sensortyp & Stromaufnahme ausgeschaltet\\
    \hline
    Lastschalter & 0.9 $\mu$A \cite{datasheet_tps2291}\\
    Feuchtigkeitssensor & 0.11 $\mu$A \cite{datasheet_hdc1000}\\
    Drucksensor & 0.1 $\mu$A \cite{datasheet_bmp280}\\
    Infrarottemperatursensor & 2 $\mu$A \cite{site_tmp007}\\
    Lichtsensor & 0.4 $\mu$A \cite{datasheet_opt3001}]\\
    \hline
    Gesamt & 3.51 $\mu$A\\
  \end{tabularx}
\end{minipage}

\subsection{aconno ACNSENSA}
Das aconno ACNSENSA besteht aus einer nRF52832 MCU und 5 Sensoren, darunter eine LSM9DS1 IMU über I2C.
Eine Auflistung der überflüssigen Komponenten, die nicht komplett abgeschaltet werden können, findet sich in Tabelle \ref{tab:cmp_acnsensa}.
Eine CR2450 Knopfzelle mit 620 mAh verliert dadurch nur $\cfrac{365d*24h}{(620mAh/0.0014mA)} \approx 2$ \% der Kapazität pro Jahr.
Der Gyrosensor der IMU verbraucht 4.0 mA bei 952 Hz.
Weitere 600 $\mu$A sind zusammen angegeben für den Beschleunigungssensor bei 952 Hz und Magnetometer bei 20 Hz, der sich aber ausschalten lässt \cite{datasheet_lsm9ds1}.
Die MCU unterstützt BT Mesh. \cite{datasheet_acnsensa}\\
Das System ist in nur geringen Stückzahlen verfügbar.
So sind bei bekannten Händler nur 5 Exemplare zu erwerben \cite{site_mouserAcnsensa}.
Die Produktwebseite sowie die Datenblätter waren über den Zeitraum dieser Arbeit teilweise nicht zugreifbar und es existieren mehrere Versionen unter dem selben Namen (vgl. Datenblatt und Produktbild bei mouser.de).

\begin{minipage}{\linewidth}
	\captionof{table}{Stromaufnahme der ausgeschalteten zusätzlichen Komponenten am aconno ACNSENSA}
	\label{tab:cmp_acnsensa}
	\begin{tabularx}{\linewidth}{X|X}
		Sensortyp & Stromaufnahme ausgeschaltet\\
		\hline
    Drucksensor & 1 $\mu$A \cite{datasheet_acnsensa}\\
    Feuchtigkeitssensor & 0.06 $\mu$A \cite{datasheet_acnsensa}\\
    Lastschalter & 0.34 $\mu$A \cite{datasheet_sip32401a}\\
    \hline
    Gesamt & 1.4 $\mu$A\\
  \end{tabularx}
\end{minipage}

\subsection{Arduino Primo Core}
Der Arduino Primo Core beinhaltet ein nRF52832 MCU und 3 Sensoren, inklusive einer LSM6DS3 IMU über I2C.
Eine Auflistung der überflüssigen Komponenten, die nicht komplett abgeschaltet werden können, findet sich in Tabelle \ref{tab:cmp_arduino_primo}.
Eine CR2032 Knopfzelle mit 230 mAh verliert dadurch $\cfrac{365d*24h}{(230mAh/0.0015mA)} \approx 5.7$ \% der Kapazität pro Jahr.
Die IMU verbraucht bei 1.6 kHz nur 1.25 mA \cite{datasheet_lsm6ds3}.
Das System kann nicht mehr gekauft werden und ist auf der Produktseite als 'retired' gelistet.\cite{site_primo} \cite{datasheet_primo}

\begin{minipage}{\linewidth}
	\captionof{table}{Stromaufnahme der ausgeschalteten zusätzlichen Komponenten am Arduino Primo Core}
	\label{tab:cmp_arduino_primo}
	\begin{tabularx}{\linewidth}{X|X}
		Sensortyp & Stromaufnahme ausgeschaltet\\
		\hline
    Magnetometer & 1 $\mu$A \cite{datasheet_lis3mdl}\\
    Feuchtigkeitssensor & 0.5 $\mu$A \cite{datasheet_hts221}\\
    \hline
    Gesamt & 1.5 $\mu$A\\
	\end{tabularx}
\end{minipage}

\section{Zusammenfassung und verwandte Arbeiten}
Mit den existierenden Lösungen wäre es grundsätzlich möglich, das Wearable umzusetzen.
Allerdings gibt es neben der problematischen Verfügbarkeit eine Limitierung auf die gegebene Hardware.
So verschwenden alle Lösungen Strom und Platz durch Sensoren, die für den benötigten Anwendungsfall nicht nötig sind, auch wenn das im Falle des aconno ACNSENSA vernachlässigbar ist.
Es ist zu beachten, dass zusätzlich zu dem Standbyverbrauch der Komponenten auch Leckströme durch Kondensatoren und Verluste durch Pull-Up Widerstände hinzuzufügen sind.\\
Alle genannten Lösungen nutzen für die Kommunikation zwischen MCU und IMU das I2C Protokoll, während in einer Arbeit \cite{comparison_i2c_spi} ermittelt wurde, dass I2C einen doppelt so hohen Energieverbrauch wie SPI hat.
Auch zwischen den IMUs gibt es Unterschiede in der Stromaufnahme.
So benötigt die IMU des Arduino Prime Core weniger als ein Drittel des Stroms der anderen Beiden und enthält als Einziges keinen Magnetometer.\\
Ein Magnetometer entspricht einem elektrischen Kompass und bestimmt die Rotation in Ausrichtung zu den Erdpolen.
In der Arbeit \cite{sensor_fusion} wurde ein Sensorfusionsalgorithmus entwickelt, der die Genauigkeit des Gyrosensors mit den Daten eines Magnetometers verbessert.
Der Algorithmus kann in zukünftigen Versionen integriert werden, falls die Genauigkeit der verwendeten IMU nicht ausreichen sollte.\\
In dem Projekt \cite{project_chordata} wird ein System zur Bewegungsaufnahme mit einer IMU entwickelt.
Hierbei werden die Sensoren per Kabel miteinander verbunden.
Es demonstriert, dass eine Bewegungserkennung mit IMUs möglich ist.
