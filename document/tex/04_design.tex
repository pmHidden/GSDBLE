%!TEX root = ../tudkom_students__201804_v1.4.tex
\chapter{Design / Concept}
\label{ch:design}
%This chapter should describe the design of the own approach on a conceptional level without mentioning the implementation details.
% 5 pages

\section{Requirements and Assumptions}
wir brauchen irgendwas zum rechnen, irgendwas mit bluetooth, irgendwelche sensoren, irgendeine stromversorgung, buttons/led oder so für user, programm und sinnvolle schnittstelle auf handy, irgendwas zum befestigen von dem ganzen

\section{System Overview}

\subsection{Bluetooth MCU}
diesdas wurde angeguckt. hat sich rausgestellt, dass bluetooth und rechner in einem am effizientesten ist. das hier sind weitere Kriterien gewesen. am ende kamen die hier infrage und das hier wurde ausgewählt, weil.

\subsection{Sensoren}
gab entweder accelerometer und gyro getrennt, oder alles in einem als imu. am ende das hier ausgesucht, weil.

\subsection{Stromversorgung}
gibt diese techniken und diese formfaktoren. das hier passt am besten.

\subsection{Android Schnittstelle}
das hier wäre ganz cool als schnittstelle

\subsection{Befestigung}
... das kommt später ...

\section{Summary}
das ist der plan. am ende haben wir dann das da. lets go
