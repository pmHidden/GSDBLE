\documentclass[longdoc,accentcolor=tud1b,11pt,paper=a4]{tudreport}
\usepackage[utf8]{inputenc}
\usepackage{textcomp}
% \usepackage{ngerman}
\usepackage[american,ngerman]{babel}
\usepackage{xspace}
\usepackage[fleqn]{amsmath} % math environments and more by the AMS
\newcounter{dummy} % necessary for correct hyperlinks (to index, bib, etc.)
\newcommand{\myfloatalign}{\centering} % how all the floats will be aligned
\renewcommand{\footerfont}{\fontfamily{\sfdefault}\fontseries{m}\fontshape{n}\footnotesize\selectfont}
\usepackage{enumitem}
	\setenumerate{noitemsep}
	\setitemize{noitemsep}
	\setdescription{noitemsep}
\usepackage{tabularx} % better tables
\setlength{\extrarowheight}{3pt} % increase table row height
\usepackage{booktabs}
\usepackage{caption}
\captionsetup{format=hang,font=small}
\usepackage[square,numbers]{natbib}
\usepackage{subfig}
\usepackage[stable,bottom]{footmisc}
\usepackage{framed}
\usepackage{color}
\usepackage{tablefootnote}
\newboolean{final} %Deklaration
\setboolean{final}{false} %Zuweisung


%======================================================
% Important information: to be set here and only here
%======================================================

% Title of the thesis in DE or EN, depending on thesis language
\newcommand{\komTitle}{Entwicklung eines Systems für die Mobile Sensordatenerfassung zur Erkennung von Ganzkörpergesten in Echtzeit\xspace}
% Translation of the title in either DE or EN, depending on thesis language
\newcommand{\komTitleTranslation}{Development of a system for mobile sensor data acquisition to recognize full body gestures in real time\xspace}

% Typ der Arbeit: Diplomarbeit Studienarbeit Master-Arbeit Bachelor-Arbeit
\newcommand{\komThesisType}{Bachelor-Arbeit\xspace}

% Studiengang: Elektrotechnik und Informationstechnik, Informationssystemtechnik, Informatik, ...
\newcommand{\komCourseOfStudy}{Informatik\xspace}

\newcommand{\komName}{Pascal Dornfeld\xspace}
\newcommand{\komSubmissionDate}{23. Juli 2019\xspace}% use only this date format

\newcommand{\komGutachter}{Gutachter: Prof. Dr.-Ing. Ralf Steinmetz\xspace}
\newcommand{\komBetreuer}{Betreuer: Philipp Müller\xspace}
\newcommand{\komExternerBetreuer}{}
\newcommand{\komID}{KOM-type-number\xspace}


%======================================================
% Setup for hyperref
%======================================================
\usepackage[pdftex,hyperfootnotes=true,pdfpagelabels]{hyperref}
	\pdfcompresslevel=9
	\pdfadjustspacing=1
\hypersetup{%
    colorlinks=false, linktocpage=false, pdfstartpage=1, pdfstartview=FitV,%
    breaklinks=true, pdfpagemode=UseNone, pageanchor=true, pdfpagemode=UseOutlines,%
    plainpages=false, bookmarksnumbered, bookmarksopen=true, bookmarksopenlevel=1,%
    hypertexnames=true, pdfhighlight=/O, %nesting=true,%frenchlinks,%
    %urlcolor=tud1b, linkcolor=tud1b, citecolor=tudtud1bccent,
    pdftitle={\komTitle, \komThesisType, \komID},%
    pdfauthor={\komName, KOM, TU Darmstadt},%
    pdfsubject={},%
    pdfkeywords={},%
    pdfcreator={},%
    pdfproducer={}%
}
\title{\komTitle}
\subtitle{\komTitleTranslation \\ \komThesisType}
\subsubtitle{\komName \\ \komID}
%\setinstitutionlogo[height]{kom_info}
\institution{\raggedleft Fachbereich Elektrotechnik \\und Informationstechnik\\%
	Fachbereich Informatik (Zweitmitglied)\\[\baselineskip]%
	Fachgebiet Multimedia Kommunikation \\%(KOM)
	Prof. Dr.-Ing. Ralf Steinmetz}
\lowertitleback{%
	Technische Universität Darmstadt \\%
	Fachbereich Elektrotechnik und Informationstechnik\\%
	Fachbereich Informatik (Zweitmitglied)\\[\baselineskip]%
	Fachgebiet Multimedia Kommunikation (KOM)\\%
	Prof. Dr.-Ing. Ralf Steinmetz%
	%Department of Electrical Engineering and Information Technology \\%
	%Department of Computer Science (Adjunct Professor) \\[\baselineskip]%
	%Multimedia Communications Lab (KOM) \\%
	%Prof. Dr.-Ing. Ralf Steinmetz %
}
\uppertitleback{%
    \textbf{\komTitle} \\%
    \komTitleTranslation \\[\baselineskip]%
	\komThesisType \\%
    Studiengang: \komCourseOfStudy \\%
	\komID \\[\baselineskip]%
	Eingereicht von \komName \\%
	Tag der Einreichung: \komSubmissionDate \\[\baselineskip]%
	\komGutachter \\%
	\komBetreuer \\%
	\komExternerBetreuer%
}

%======================================================
% MAIN DOCUMENT STARTS HERE
%======================================================
\begin{document}

	\colorlet{tudidentbar}{tud1b} %first page colored - DO NOT MODIFY THIS
	%======================================================
	% The front matter
	%======================================================
	\pagenumbering{roman}
	\frenchspacing
	\raggedbottom
	\selectlanguage{ngerman} % american ngerman
	\maketitle

	% identbar color for the rest of the thesis - DO NOT MODIFY THIS
	\colorlet{tudidentbar}{tud0b}

    \begin{otherlanguage}{ngerman}

    \chapter*{Erklärung zur Abschlussarbeit gemäß § 23 Abs.\ 7 APB der TU Darmstadt}
    Hiermit versichere ich, \komName, die vorliegende \komThesisType ohne Hilfe Dritter und nur mit den angegebenen Quellen und Hilfsmitteln angefertigt zu haben.
    Alle Stellen, die Quellen entnommen wurden, sind als solche kenntlich gemacht worden.
    Diese Arbeit hat in gleicher oder ähnlicher Form noch keiner Prüfungsbehörde vorgelegen.\\

    \noindent Mir ist bekannt, dass im Falle eines Plagiats (§38 Abs.2 APB) ein Täuschungsversuch vorliegt, der dazu führt, dass die Arbeit mit 5,0 bewertet und damit ein Prüfungsversuch verbraucht wird.
    Abschlussarbeiten dürfen nur einmal wiederholt werden.\\

    \noindent Bei der abgegebenen \komThesisType stimmen die schriftliche und die zur Archivierung eingereichte elektronische Fassung überein.

    \vspace{4em}

    \noindent Darmstadt, den \komSubmissionDate

    \vspace{3em}

    \noindent\rule{5cm}{0.4pt}

    \noindent\komName

    \end{otherlanguage}

	\tableofcontents
	%\listoffigures
	%\listoftables

	%======================================================
	% The main matter (insert your contents here)
	%======================================================
	\cleardoublepage
	\pagenumbering{arabic}

	\begin{abstract}
	 The abstract goes here...
\end{abstract}
	%*****************************************
\chapter{Introduction}
%*****************************************
\hint{This chapter should motivate the thesis, provide a clear description of the problem to be solved, and describe the major contributions of this thesis. The chapter should have a length of about two pages!}

\section{Motivation}
What is the motivation for doing research in this area?

\section{Problem Statement and Contribution}
What is the problem that should be solved with this thesis?

\section{Outline}
How is the rest of this thesis structured?
	%*****************************************
\chapter{Background}
\label{ch:background}
%*****************************************
\hint{This chapter should give a comprehensive overview on the background necessary to understand the thesis.
The chapter should have a length of about five pages!}
Bib\TeX-Test: \cite{Steinmetz2005} \citeauthor{Steinmetz2005} \citep{Steinmetz2005}

\section{Background Topic 1}

\section{Background Topic 2}

\section{Summary}
	%*****************************************
\chapter{Related Work}
\label{ch:relatedwork}
%*****************************************
\hint{This chapter should give a comprehensive overview on the related work done by other authors followed by an analysis why the existing related work is not capable of solving the problem described in the introduction.
The chapter should have a length of about three to five pages!}
\section{Related Work Area 1}

\section{Related Work Area 2}

\section{Analysis of Related Work}

\section{Summary}
	%*****************************************
\chapter{Design / Concept}
\label{ch:design}
%*****************************************
\hint{This chapter should describe the design of the own approach on a conceptional level without mentioning the implementation details. The section should have a length of about five pages.}

\section{Requirements and Assumptions}
we need: microcontroller, (wireless) connection to phone, 6-or-more-axis sensor, power converter, battery or something-like-that
for evaluation: something to measure power consumption

\section{System Overview}

\subsection{Component 1}

\subsection{Component 2}

\section{Summary}
	%*****************************************
\chapter{Implementation}
\label{ch:implementation}
%*****************************************

\hint{This chapter should describe the details of the implementation addressing the following questions: \\ \\
1. What are the design decisions made? \\
2. What is the environment the approach is developed in? \\
3. How are components mapped to classes of the source code? \\
4. How do the components interact with each other?  \\
5. What are limitations of the implementation? \\ \\
The section should have a length of about five pages.}
\section{Design Decisions}

\section{Architecture}

\section{Interaction of Components}

\section{Summary}
	%*****************************************
\chapter{Evaluation}
\label{ch:evaluation}
%*****************************************
\hint{This chapter should describe how the evaluation of the implemented mechanism was done. \\ \\
1. Which evaluation method is used and why? Simulations, prototype? \\
2. What is the goal of the evaluation? Comparison? Proof of concept? \\
3. Wich metrics are used for characterizing the performance, costs, fairness, and efficiency of the system?\\
4. What are the parameter settings used in the evaluation and why? If possible always justify why a certain threshold has been chose for a particular parameter.  \\
5. What is the outcome of the evaluation? \\ \\
The section should have a length of about five to ten pages.}
\section{Goal and Methodology}

\section{Evaluation Setup}

\section{Evaluation Results}

\section{Analysis of Results}
	%*****************************************
\chapter{Conclusions}
\label{ch:closure}
%*****************************************

\hint{This chapter should summarize the thesis and describe the main contributions of the thesis. Subsequently, it should describe possible future work in the context of the thesis. What are limitations of the developed solutions? Which things can be improved?
The section should have a length of about three pages.}

\section{Summary}

\section{Contributions}

\section{Future Work}

\section{Final Remarks}


	%======================================================
	% The back matter
	%======================================================
	%\cleardoublepage
	\refstepcounter{dummy}
	\addcontentsline{toc}{chapter}{\bibname}
	\bibliographystyle{alpha} % <--- layout of the bib
	\bibliography{biblio} % file name of your bib

\end{document}
%======================================================
%======================================================
